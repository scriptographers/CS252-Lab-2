\documentclass[11pt, fleqn]{article}

\usepackage{amsmath}
\usepackage{amssymb}
\usepackage{amsthm}
\usepackage{mathtools}
\usepackage{hyperref}
\usepackage{ulem}
\usepackage{enumitem}
\usepackage[left=0.75in, right=0.75in, bottom=0.75in, top=1.0in]{geometry}
\usepackage{floatrow}
\usepackage{graphicx}
\usepackage[export]{adjustbox}
\usepackage{sectsty}
\sectionfont{\centering}

\usepackage[perpage]{footmisc}

\usepackage{fancyhdr}
\pagestyle{fancy}
\fancyhf{}
\lhead{190100036 \& 190100044 \& 190100055}
\rhead{CS 252: Lab 2}
\renewcommand{\footrulewidth}{1.0pt}
\cfoot{Page \thepage}

\setlength{\parindent}{0em}
\renewcommand{\arraystretch}{2}

\title{CS 252: Lab 2}
\author{
\begin{tabular}{|c|c|c|}
     \hline
     Krushnakant Bhattad & Devansh Jain & Harshit Varma \\
     \hline
     190100036 & 190100044 & 190100055 \\
     \hline
\end{tabular}
}
% \author{
%   Krushnakant Bhattad\\
%   190100036
%   \and
%   Devansh Jain\\
%   190100044
%   \and
%   Harshit Varma\\
%   190100055
% }
\date{\today}

\usepackage{hyperref}

\usepackage[dvipsnames]{xcolor}

\begin{document}

\maketitle
\tableofcontents
\thispagestyle{empty}
\setcounter{page}{0}
\renewcommand{\arraystretch}{1}

\vspace{5em}
\section*{About the report}


\newpage 
\section*{(i) \texttt{ifconfig}}
\label{parta}
\addcontentsline{toc}{section}{(i) \texttt{ifconfig}}
\setcounter{equation}{0}

\textbf{Network Interface Information:}
\begin{itemize}
 
    \item Team Member: Krushnakant Bhattad
    \begin{itemize}[itemsep=-0.5ex]
    \item Type: \texttt{wifi}
    \item IPv4: \texttt{192.168.43.1    }
    \item IPv6: \texttt{fe80::d85a:33ff:fef0:2d33 }  
    \item MAC: \texttt{40:5b:d8:9b:c0:4d}
    \item MTU: \texttt{1500}
    \item Transmit Queue Length: \texttt{1000}
    \end{itemize}
    
    \item Team Member: Devansh Jain
    \begin{itemize}[itemsep=-0.5ex]
        \item Type: \texttt{wifi}
        \item IPv4: \texttt{192.168.29.249}
        \item IPv6: \texttt{fe80::113d:1e1c:c608:4f93}
        \item MAC: \texttt{dc:71:96:3c:31:ba}
        \item MTU: \texttt{1500}
        \item Transmit Queue Length: \texttt{1000}
    \end{itemize}
    
    \item Team Member: Harshit Varma
    \begin{itemize}[itemsep=-0.5ex]
    \item Type: \texttt{Ethernet}\footnote{WSL vEthernet adapter: WSL-2 has a virtualized ethernet adapter with its own unique addresses}
        \item IPv4: \texttt{172.23.207.194}
        \item IPv6: \texttt{fe80::215:5dff:febd:ce13}
        \item MAC: \texttt{00:15:5d:bd:ce:13}
        \item MTU: \texttt{1500}
        \item Transmit Queue Length: \texttt{1000}
    \end{itemize}

\end{itemize}

\hrulefill
\medskip

\textbf{Number of bits used for:}
\begin{itemize}[itemsep=-1ex]
    \item IPv4: \texttt{32}
    \item IPv6: \texttt{128}
    \item MAC: \texttt{48}
\end{itemize}

\hrulefill
\medskip 

MTU stands for \textbf{Maximum Transmission Unit}, which defines the size of the largest protocol data unit (PDU) that can be communicated in a single network layer (layer-3) transaction. It is specified in terms of \textbf{bytes} or octets of the largest PDU that the layer can pass onwards. \footnote{Source:  \href{https://en.wikipedia.org/wiki/Maximum_transmission_unit}{Wikipedia} }

\hrulefill
\medskip

Units for Transmit Length Queue: \textbf{packets}


\newpage
\section*{(ii) \texttt{traceroute <dest host IP or name>}}
\label{partb}
\addcontentsline{toc}{section}{(ii) \texttt{traceroute <dest host IP or name>}}
\setcounter{equation}{0}

Team Member: Harshit Varma\\
Destination: \texttt{www.google.com}

\hrulefill
\smallskip

Source: Local machine\\
Number of hops: \texttt{8}\\
Average RTT: \texttt{7.33}\\
Destination IP: \texttt{172.17.112.70} (Mumbai) \footnote{\url{https://ipinfo.io/} was used for getting location from IP address, equivalently, the \texttt{whois} command can also be used}

\hrulefill
\smallskip

Source: Tokyo (Japan, Asia)\\
Number of hops: \texttt{5}\\
Average RTT: $<$\texttt{1}\\
Destination IP: \texttt{172.217.25.100} (Mountain View, California)

\hrulefill
\smallskip

Source: Sydney (Australia)\\
Number of hops: \texttt{10}\\
Average RTT: \texttt{1}\\
Destination IP: \texttt{142.250.66.196} (Sydney)

\hrulefill
\smallskip

Source: Oslo (Norway, Europe)\\
Number of hops: \texttt{11}\\
Average RTT: \texttt{9.33}\\
Destination IP: \texttt{142.250.74.36} (Mountain View, California)

\hrulefill
\smallskip

Source: Buenos Aires (Argentina, South America)\\
Number of hops: \texttt{9}\\
Average RTT: \texttt{6.67}\\
Destination IP: \texttt{172.217.172.228} (Buenos Aires)

\hrulefill
\smallskip

Source: Los Angeles (USA, North America)\\
Number of hops: \texttt{6}\\
Average RTT: \texttt{1}\\
Destination IP: \texttt{216.58.193.196} (Mountain View, California)

\hrulefill
\medskip

The destination IP addresses changes when the source changes, even if the domain name is the same.

\smallskip

In fact, it even changes with time keeping the source and the destination fixed.\\ 
For example, when \texttt{traceroute} was run again from the local machine, $\approx 10$ mins later, the destination IP address changed from \texttt{216.58.203.36} to \texttt{142.250.76.164} (also Mumbai) and then after $\approx 5$ mins, it reverted back to \texttt{216.58.203.36}, and then it changed to \texttt{216.58.200.196} (Zhengzhou, China), approximately $2$ mins later.

\smallskip

This is because when a client attempts to connect to Google, several DNS servers resolve \texttt{www.google.com} into multiple IP addresses via \textbf{Round Robin policy}. This acts as the first level of load balancing and directs the client to different Google cluster, which has thousands of server. \footnote{Reference: \url{https://en.wikipedia.org/wiki/Google_data_centers}}

\newpage
Team Member: Devansh Jain\\
Destination: \texttt{www.cnn.com}

\hrulefill
\smallskip

Source: Local machine\\
Number of hops: \texttt{12}\\
Average RTT: \texttt{77.5}\\
Destination IP: \texttt{151.101.65.67} () \footnote{\url{https://ipinfo.io/} was used for getting location from IP address, equivalently, the \texttt{whois} command can also be used}

\hrulefill
\smallskip

Source: Tokyo (Japan, Asia)\\
Number of hops: \texttt{6}\\
Average RTT: $<$\texttt{1}\\
Destination IP: \texttt{151.101.1.67} (San Francisco, California)

\hrulefill
\smallskip

Source: Sydney (Australia)\\
Number of hops: \texttt{10}\\
Average RTT: \texttt{<1}\\
Destination IP: \texttt{151.101.29.67} (Sydney)

\hrulefill
\smallskip

Source: Oslo (Norway, Europe)\\
Number of hops: \texttt{7}\\
Average RTT: \texttt{2}\\
Destination IP: \texttt{151.101.1.67} (San Francisco, California)

\hrulefill
\smallskip

Source: Buenos Aires (Argentina, South America)\\
Number of hops: \texttt{11}\\
Average RTT: \texttt{152}\\
Destination IP: \texttt{151.101.5.67} ()

\hrulefill
\smallskip

Source: Los Angeles (USA, North America)\\
Number of hops: \texttt{5}\\
Average RTT: \texttt{1.33}\\
Destination IP: \texttt{151.101.25.67} ()

\hrulefill


\newpage
\section*{(iii) \texttt{ping <dest host>}}
\label{partc}
\addcontentsline{toc}{section}{(iii) \texttt{ping <dest host>}}
\setcounter{equation}{0}

Team Member: Harshit Varma

\hrulefill
\smallskip

Server: \texttt{www.nasa.gov} \\
Expected Location: \texttt{USA, North America}\\
Average RTT using local machine: \texttt{8.433} (15 transmitted, 15 received)\\
Average RTT using \texttt{ping.eu}: \texttt{27.153} (4 transmitted, 4 received)

\hrulefill
\smallskip

Server: \texttt{www.gov.uk}\\
Expected Location: \texttt{UK, Europe}\\
Average RTT using local machine: \texttt{9.561} (15 transmitted, 15 received)\\
Average RTT using \texttt{ping.eu}: \texttt{32.261} (4 transmitted, 4 received)

\hrulefill
\smallskip

Server: \texttt{www.gov.au}\\
Expected Location: \texttt{Australia}\\
Average RTT using local machine: \texttt{24.009} (15 transmitted, 15 received)\\
Average RTT using \texttt{ping.eu}: \texttt{7.797} (4 transmitted, 4 received)

\hrulefill
\smallskip

Server: \texttt{www.statehouse.gov.ng}\\
Expected Location: \texttt{Nigeria, Africa}\\
Average RTT using local machine: \texttt{269.625} (34 transmitted, 16 received)\\
Average RTT using \texttt{ping.eu}: \texttt{151.545} (4 transmitted, 4 received)

\hrulefill
\smallskip

Server: \texttt{www.gob.cl}\\
Expected Location: \texttt{Chile, South America}\\
Average RTT using local machine: \texttt{17.144} (15 transmitted, 15 received)\\
Average RTT using \texttt{ping.eu}: \texttt{4.753} (4 transmitted, 4 received)

\hrulefill
\smallskip

Server: \texttt{www.jnto.go.jp}\\
Expected Location: \texttt{Japan, Asia}\\
Average RTT using local machine: \texttt{24.113} (15 transmitted, 15 received)\\
Average RTT using \texttt{ping.eu}: \texttt{26.369} (4 transmitted, 4 received)

\hrulefill
\smallskip

Server: \texttt{comnap.aq}\footnote{Tried to ping \texttt{www.ats.aq, www.soki.aq, www.soos.aq} but no packets were returned}\\
Expected Location: \texttt{Antarctica \footnote{A \texttt{traceroute} reveals that the actual location is Christchurch, New Zealand}}\\
Average RTT using local machine: \texttt{189.067} (15 transmitted, 15 received)\\
Average RTT using \texttt{ping.eu}: \texttt{295.686} (4 transmitted, 4 received)



\newpage


Team Member: Devansh Jain

\hrulefill
\smallskip

Server: \texttt{www.canada.ca} \\
Expected Location: \texttt{Canada, North America}\\
Average RTT using local machine: \texttt{13.441} (20 transmitted, 20 received)\\
Average RTT using \texttt{ping.eu}: \texttt{35.974} (4 transmitted, 4 received)

\hrulefill
\smallskip

Server: \texttt{www.spth.gob.es}\\
Expected Location: \texttt{Spain, Europe}\\
Average RTT using local machine: \texttt{182.319} (20 transmitted, 20 received)\\
Average RTT using \texttt{ping.eu}: \texttt{12.891} (4 transmitted, 4 received)

\hrulefill
\smallskip

Server: \texttt{www.australia.gov.au}\\
Expected Location: \texttt{Australia}\\
Average RTT using local machine: \texttt{24.133} (20 transmitted, 20 received)\\
Average RTT using \texttt{ping.eu}: \texttt{27.464} (4 transmitted, 4 received)

\hrulefill
\smallskip

Server: \texttt{www.gov.za}\\
Expected Location: \texttt{South Africa, Africa}\\
Average RTT using local machine: \texttt{360.516} (20 transmitted, 20 received)\\
Average RTT using \texttt{ping.eu}: \texttt{185.209} (4 transmitted, 4 received)

\hrulefill
\smallskip

Server: \texttt{www.argentina.gob.ar}\\
Expected Location: \texttt{Argentina, South America}\\
Average RTT using local machine: \texttt{178.347} (20 transmitted, 20 received)\\
Average RTT using \texttt{ping.eu}: \texttt{5.219} (4 transmitted, 4 received)

\hrulefill
\smallskip

Server: \texttt{www.nepal.gov.np}\\
Expected Location: \texttt{Nepal, Asia}\\
Average RTT using local machine: \texttt{301.342} (20 transmitted, 20 received)\\
Average RTT using \texttt{ping.eu}: \texttt{163.905} (4 transmitted, 4 received)

\hrulefill
\smallskip

Server: \texttt{www.comnap.aq}\footnote{Tried to ping \texttt{www.ats.aq, www.soki.aq, www.soos.aq} but no packets were returned}\\
Expected Location: \texttt{Antarctica \footnote{A \texttt{traceroute} reveals that the actual location is Christchurch, New Zealand}}\\
Average RTT using local machine: \texttt{252.765} (20 transmitted, 20 received)\\
Average RTT using \texttt{ping.eu}: \texttt{294.773} (4 transmitted, 4 received)



\newpage


Team Member: Krushnakant Bhattad

\hrulefill
\smallskip

Server: \texttt{www.gob.mx} \\
Expected Location: \texttt{Mexico, North America}\\
Average RTT using local machine: \texttt{38.448 ms} (20 transmitted, 20 received)\\
Average RTT using \texttt{ping.eu}: \texttt{30.273 ms} (4 transmitted, 4 received)


\hrulefill
\smallskip

Server: \texttt{www.admin.ch} \\
Expected Location: \texttt{Switzerland, Europe}\\
Average RTT using local machine: \texttt{255.733 ms} (20 transmitted, 20 received)\\
Average RTT using \texttt{ping.eu}: \texttt{27.594 ms} (4 transmitted, 4 received)



\hrulefill
\smallskip

Server: \texttt{www.govt.nz} \\
Expected Location: \texttt{New Zealand}\\
Average RTT using local machine: \texttt{26.581 ms} (20 transmitted, 20 received)\\
Average RTT using \texttt{ping.eu}: \texttt{192.383 ms} (4 transmitted, 4 received)


\hrulefill
\smallskip

Server: \texttt{www.pm.gov.tn} \footnote{Tried to ping \texttt{www.grnnet.gov.na, www.presidence.gov.mg, www.president.go.ke,} etc. but no packets were returned}\\
Expected Location: \texttt{Tunisia, Africa}\\
Average RTT using local machine: \texttt{253.009 ms} (20 transmitted, 20 received)\\
Average RTT using \texttt{ping.eu}: \texttt{46.893 ms} (4 transmitted, 4 received)


\hrulefill
\smallskip


Server: \texttt{www.ttconnect.gov.tt} \\
Expected Location: \texttt{Trinidad and Tobago, South America}\\
Average RTT using local machine: \texttt{393.735 ms} (20 transmitted, 20 received)\\
Average RTT using \texttt{ping.eu}: \texttt{159.213 ms} (4 transmitted, 4 received)


\hrulefill
\smallskip


Server: \texttt{www.gco.gov.qa} \\
Expected Location: \texttt{Qatar, Asia}\\
Average RTT using local machine: \texttt{22.229 ms} (20 transmitted, 20 received)\\
Average RTT using \texttt{ping.eu}: \texttt{5.193 ms} (4 transmitted, 4 received)




\hrulefill
\smallskip


Server: \texttt{www.acap.aq} \\
Expected Location: \texttt{Antarctica}\footnote{A \texttt{traceroute} reveals that the actual location is Sydney, 
Australia}\\
Average RTT using local machine: \texttt{212.604 ms} (20 transmitted, 20 received)\\
Average RTT using \texttt{ping.eu}: \texttt{248.566 ms} (4 transmitted, 4 received)



\newpage
\section*{(iv) \texttt{Iperf}}
\label{partd}
\addcontentsline{toc}{section}{(iv) \texttt{Iperf}}
\setcounter{equation}{0}

Team Member: Harshit Varma

\hrulefill
\smallskip

% Command used: iperf3 -c <server_name> -u -b <value>M

Server: \texttt{iperf.scottlinux.com} (California, USA)\\
TCP Bandwidth:\\
Sender: \texttt{51.0 Mbits/sec}\\
Reciever: \texttt{48.9 Mbits/sec}

For UDP:\\
The observed throughput becomes much less than the specified UDP rate is \texttt{128 Mbits/sec}

\hrulefill
\smallskip

Server: \texttt{speedtest.hostkey.ru} \footnote{Reference for the iperf server names: \url{https://iperf.cc/}} (Moscow, Russia)\\
TCP Bandwidth:\\
Sender: \texttt{52.1 Mbits/sec}\\
Reciever: \texttt{52.1 Mbits/sec}

\hrulefill
\smallskip

Server: \texttt{iperf.biznetnetworks.com} (Jakarta, Indonesia)\\
TCP Bandwidth:\\
Sender: \texttt{1.79 Mbits/sec}\\
Reciever: \texttt{1.41 Mbits/sec}

\hrulefill
\medskip

The ``Sender" is the client, the value mentioned is the upload speed from the client to the \texttt{iperf} server.\\
The ``Receiver" is \texttt{iperf} server, the value mentioned is the download speed on \texttt{iperf} server for the client.\footnote{Reference: \url{https://github.com/esnet/iperf/issues/480#issuecomment-307205313}}

\medskip

I was often getting the following errors while trying to run \texttt{iperf}/\texttt{iperf3}:
\begin{itemize}[itemsep=-0.5ex]

    \item \texttt{iperf3: error - the server is busy running a test. try again later}\\
    Changing the port number using \texttt{-p} flag sometimes seemed to resolve this issue.\\
    Sometimes, running \texttt{iperf} instead of \texttt{iperf3} resolved this issue.
    
    \item \texttt{connect failed: No route to host}\\
    This was observed while using \texttt{iperf}, couldn't find any fix for this, the same server worked using \texttt{iperf3} though
    
    \item \texttt{iperf3: error - unable to connect to server: Connection refused}\\
    Got this error when trying to use different ports
    
\end{itemize}

\newpage
\section*{Appendix}
\addcontentsline{toc}{section}{Appendix}
\setcounter{equation}{0}


\end{document}